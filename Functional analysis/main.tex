\documentclass[12pt, letterpaper, twoside]{article}
\usepackage{enumitem}
\setlist{nolistsep}
%Russian-specific packages
%--------------------------------------
\usepackage[T2A]{fontenc}
\usepackage[utf8]{inputenc}
\usepackage[russian]{babel}
%--------------------------------------
 
%Hyphenation rules
%--------------------------------------
\usepackage{hyphenat}
\hyphenation{ма-те-ма-ти-ка вос-ста-нав-ли-вать}
%--------------------------------------

\usepackage{amssymb}
\usepackage{amsmath}
\usepackage{latexsym}
 
\begin{document}
 
\begin{center}
    {\LARGE Функциональный анализ}
    \vspace*{1.5cm}
    
    {\Large Полные метрические пространства}
\end{center}

\noindent
\textbf{Опр:} \((X, \rho)\) --- \textit{метрическое пространство}, если \(X\) --- множество,\\
\(\rho: X \times X \rightarrow \mathbb{R}\) --- \textit{метрика}, и выполняются следующие условия:
\begin{itemize}
    \item \(\rho(x, y) \geq 0, \qquad  \forall x, y \in X\)
    \item \(\rho(x, y) = 0 \enspace \Leftrightarrow \enspace x = y, \qquad  \forall x, y \in X\)
    \item \(\rho(x, y) = \rho(y, x), \qquad  \forall x, y \in X\)
    \item \(\rho(x, y) \leq \rho(x, z) + \rho(z, y), \qquad  \forall x, y, z \in X\)
\end{itemize} 

\vspace*{0.3cm}
\noindent
\textbf{Опр:} \(\{B_r(x)\}_{r>0}\) --- \textit{база топологии} (т.е. семейство открытых подмножеств, через которые любой элемент представим в виде их объединения), где \(B_r(x) = \{y \in X : \rho(x, y) < r\}\) --- \textit{открытый шар} \(,\enspace r > 0, \enspace x \in X\)
  
\vspace*{0.3cm}
\noindent
\textbf{Опр:} \(U\) --- \textit{открытое} множество, если \(\forall x \in U \enspace \exists r > 0: B_r(x) \subset U\)

\vspace*{0.3cm}
\noindent
\textbf{Опр:} \(\{B_{r_n} (x)\}_{r_n \in \mathbb{Q}}\) --- \textit{счётная} база в \(X\)

\vspace*{0.3cm}
\noindent
\textbf{Опр:} \(A \subset X, \enspace A\) --- \textit{замкнутое} \enspace \Leftrightarrow \enspace \(X \smallsetminus A\) --- открытое (или \(\enspace \forall \enspace \{x_n\}^\infty _{n = 1} : \enspace x_n \in A \enspace \exists \lim_{n \rightarrow \infty} x_n = x_0 \Rightarrow x_0 \in A\))

\vspace*{0.3cm}
\noindent
\textbf{Опр:} \(D_r(x) = \{y \in X: \rho (x, y) \leq r \}\) --- \textit{замкнутый круг}
    
\vspace*{0.4cm}
\noindent
\[\lim_{n \rightarrow \infty} x_n = x_0 \enspace \Leftrightarrow \enspace \lim_{n \rightarrow \infty} \rho (x_n, x_0) = 0\]

\vspace*{0.3cm}
\noindent
\textbf{Опр:} \(\{x_n\}^{\infty}_{n = 1}\) --- \textit{фундаментальная последовательность} в \(X\), если \(\forall \epsilon > 0 \enspace \exists N \in \mathbb{N} : n, m > N \Rightarrow \rho (x_n, x_m) < \epsilon\) 
    
\vspace*{0.3cm}
\noindent
\textbf{Свойство:} \((x, \rho)\) --- метрическое пр-во, \(\{x_n\}^{\infty}_{n = 1}, \enspace x_n \in X\) \\
\(\exists \lim_{n \rightarrow \infty} x_n = a \enspace \Rightarrow \enspace \{x_n\}^{\infty}_{n = 1}\) --- фундаментальная последовательность

\vspace*{0.3cm}
\noindent
\textbf{Опр:} \((X, \rho)\) --- метр. пр-во, \(X\) --- \textit{полное}, если \(\forall \enspace \{x_n\}\) --- фунд. \(\enspace \Rightarrow \enspace \exists \lim_{n \rightarrow \infty} x_n = a \in X\)

\vspace*{0.3cm}
\noindent
\textbf{Опр:} \(A \in X, (X, \rho), A\) --- \textit{ограниченное}, если \(\exists x_0 \in X, R > 0: A \subset B_R(x_0)\)

\vspace*{0.3cm}
\noindent
\textbf{Теорема} (св-ва фунд. посл-ти): \\
\((X, \rho)\) --- метрическое пр-во, \(\{x_n\}^{\infty}_{n = 1}\) --- фунд. пос-ть \(\Rightarrow\)
\begin{enumerate}
    \item \(\{x_n\}^{\infty}_{n = 1}\) --- ограниченная, т. е. \(\exists a \in X, R > 0 : x_n \in B_R(a) \enspace \forall n \in \mathbb{N}\)
    \item \(\exists \{x_{n_k}\}^{\infty}_{k = 1}\) --- подп-ть \(\{x_n\}^{\infty}_{n = 1}: \exists \lim_{k \rightarrow \infty} x_{n_k} = a \enspace \Rightarrow \enspace \lim_{n \rightarrow \infty} x_n = a\)
    \item \(\{\epsilon_k\}^{\infty}_{k = 0}, \epsilon > 0 \enspace \Rightarrow \enspace \exists\) подпос-ть \(\{x_{n_k}\}: \forall j > k \in \mathbb{N} \enspace \rho (x_{n_k}, x_{n_j}) < \epsilon\)
\end{enumerate}

\begin{center}
    \vspace*{1.5cm}
    {\Large Б\.aнаховы пространства}
\end{center}

\noindent
\textbf{Опр:} \(X\) --- линейное пр-во над полем \(k \ (k = \mathbb{R} \ || \ k = \mathbb{C})\); \enspace \(p: X \rightarrow \mathbb{R}, \\ p\) --- \textit{полунорма}, если:
\begin{enumerate}
    \item \(p(x + y) \leq p(x) + p(y), \qquad \forall x, y \in X\)
    \item \(p(\lambda x) = |\lambda| p(x), \qquad \forall x \in X, \lambda \in k \)
\end{enumerate}

\vspace*{0.3cm}
\noindent
\textbf{Свойство} (полунормы): \\
\(X\) --- лин. пр-во, \(p\) --- полунорма \ \Rightarrow \ \(p(\mathbb{O}) = 0, \ p(x) = p(-x), \ p(x) \geq 0 \enspace \forall x \in X\)

\vspace*{0.3cm}
\noindent
\textbf{Опр:} \(X\) --- лин. пр-во над \(k\), \ \(p\) --- \textit{норма}, если \(p(x) = 0 \ \Leftrightarrow \ x = \mathbb{O} \ \in X \)

\vspace*{0.3cm}
\noindent
\textbf{Опр:} \( (X, || \cdot ||) \) --- \textit{нормированное} пр-во; \(\rho(x, y) := ||x - y||\); \ (аксиомы нормы \( \ \Rightarrow \ \) аксиомы метрики)

\vspace*{0.3cm}
\noindent
\textbf{Опр:} \( (X, || \cdot ||) \) --- \textit{банахово} пр-во, если \( (X, \rho) \) --- полное

\vspace*{0.3cm}
\noindent
\textbf{Опр:} 
\begin{enumerate}
    \item \(X\) --- лин. пр-во над \(k \ (k = \mathbb{R} \ || \ k = \mathbb{C})\); \ \(L \subset X\), \(L\) --- подпространство \textit{в алгебраическом смысле}, если \(L\) --- лин. подпр-во над \(k\), т.е. \( \forall \ \alpha, \beta \in k, \ x, y \in L \ \Rightarrow \ \alpha x + \beta y \in L \)
    \item \( (X, || \cdot ||) \) --- лин. нормир. пр-во, \( L \subset X, L \) --- \textit{подпространство}, если \(L\) --- подпр-во в алгебр. смысле и замкнуто 
\end{enumerate}

\vspace*{0.3cm}
\noindent
\textbf{Опр:} \( (X, || \cdot ||), \enspace \{x_k\}^{\infty}_{k = 1}, \enspace x_k \in X, \enspace S_n = \sum_{k=1}^n x_k \)
\vspace*{0.1cm}
\begin{enumerate}
    \item \( \sum_{k=1}^\infty x_k \) --- \textit{сходится}, если \( \exists \lim_{n \rightarrow \infty} S_n = S, \enspace S \in X, S = \sum_{k=1}^\infty x_k \)
    \vspace*{0.1cm}
    \item \( \sum_{k=1}^\infty x_k \) --- \textit{сходится абсолютно}, если \( \sum_{k=1}^\infty ||x_k|| \) --- сходится
\end{enumerate}

\vspace*{0.3cm}
\noindent
\textbf{Теорема} (Критерий полноты нормированного пространства): \\
\( (X, || \cdot ||), \ X \) --- полное \( \enspace \Leftrightarrow \enspace \) из абсолютной сходимости \( \ \Rightarrow \ \) сходимость

\begin{center}
    \vspace*{1.5cm}
    {\Large Пространство ограниченных функций}
\end{center}

\vspace*{0.3cm}
\noindent
\textbf{Опр:} \(A\) --- мн-во, \( \ m(A) = \{ f: A \rightarrow \mathbb{R} (|| \ \mathbb{C}) \ | \ f \) --- огр. (т.е. \( \sup_{x \in A} |f(x)| \leq + \infty) \} \) --- мн-во всех ограниченных ф-ий на \(A\); \( ||f||_{\infty} = sup_{x \in A} |f(x)| \)

\vspace*{0.3cm}
\noindent
\textbf{Теорема:} \( m(A) \) --- банахово пр-во
    
\end{document}
